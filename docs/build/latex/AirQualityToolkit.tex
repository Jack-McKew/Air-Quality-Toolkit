%% Generated by Sphinx.
\def\sphinxdocclass{report}
\documentclass[letterpaper,10pt,english,openany,oneside]{sphinxmanual}
\ifdefined\pdfpxdimen
   \let\sphinxpxdimen\pdfpxdimen\else\newdimen\sphinxpxdimen
\fi \sphinxpxdimen=.75bp\relax

\PassOptionsToPackage{warn}{textcomp}
\usepackage[utf8]{inputenc}
\ifdefined\DeclareUnicodeCharacter
 \ifdefined\DeclareUnicodeCharacterAsOptional
  \DeclareUnicodeCharacter{"00A0}{\nobreakspace}
  \DeclareUnicodeCharacter{"2500}{\sphinxunichar{2500}}
  \DeclareUnicodeCharacter{"2502}{\sphinxunichar{2502}}
  \DeclareUnicodeCharacter{"2514}{\sphinxunichar{2514}}
  \DeclareUnicodeCharacter{"251C}{\sphinxunichar{251C}}
  \DeclareUnicodeCharacter{"2572}{\textbackslash}
 \else
  \DeclareUnicodeCharacter{00A0}{\nobreakspace}
  \DeclareUnicodeCharacter{2500}{\sphinxunichar{2500}}
  \DeclareUnicodeCharacter{2502}{\sphinxunichar{2502}}
  \DeclareUnicodeCharacter{2514}{\sphinxunichar{2514}}
  \DeclareUnicodeCharacter{251C}{\sphinxunichar{251C}}
  \DeclareUnicodeCharacter{2572}{\textbackslash}
 \fi
\fi
\usepackage{cmap}
\usepackage[T1]{fontenc}
\usepackage{amsmath,amssymb,amstext}
\usepackage{babel}
\usepackage{times}
\usepackage[Bjarne]{fncychap}
\usepackage{sphinx}

\usepackage{geometry}

% Include hyperref last.
\usepackage{hyperref}
% Fix anchor placement for figures with captions.
\usepackage{hypcap}% it must be loaded after hyperref.
% Set up styles of URL: it should be placed after hyperref.
\urlstyle{same}
\addto\captionsenglish{\renewcommand{\contentsname}{Contents:}}

\addto\captionsenglish{\renewcommand{\figurename}{Fig.}}
\addto\captionsenglish{\renewcommand{\tablename}{Table}}
\addto\captionsenglish{\renewcommand{\literalblockname}{Listing}}

\addto\captionsenglish{\renewcommand{\literalblockcontinuedname}{continued from previous page}}
\addto\captionsenglish{\renewcommand{\literalblockcontinuesname}{continues on next page}}

\addto\extrasenglish{\def\pageautorefname{page}}

\setcounter{tocdepth}{1}



\title{Air Quality Toolkit Documentation}
\date{Oct 05, 2018}
\release{1}
\author{Jack McKew}
\newcommand{\sphinxlogo}{\vbox{}}
\renewcommand{\releasename}{Release}
\makeindex

\begin{document}

\maketitle
\sphinxtableofcontents
\phantomsection\label{\detokenize{index::doc}}



\chapter{Stitcher}
\label{\detokenize{index:stitcher}}\label{\detokenize{index:welcome-to-air-quality-toolkit-s-documentation}}\label{\detokenize{index:module-Stitcher}}\index{Stitcher (module)}\index{Stitcher() (in module Stitcher)}

\begin{fulllineitems}
\phantomsection\label{\detokenize{index:Stitcher.Stitcher}}\pysiglinewithargsret{\sphinxcode{\sphinxupquote{Stitcher.}}\sphinxbfcode{\sphinxupquote{Stitcher}}}{\emph{filepath\_list}, \emph{filename}, \emph{scale}, \emph{headers\_list}, \emph{output\_filename}}{}
This function sums multiple dataframes together and outputs a csv of the result.
\begin{quote}\begin{description}
\item[{Parameters}] \leavevmode\begin{itemize}
\item {} 
\sphinxstyleliteralstrong{\sphinxupquote{filepath\_list}} (\sphinxhref{https://docs.python.org/3/library/stdtypes.html\#list}{\sphinxstyleliteralemphasis{\sphinxupquote{list}}}\sphinxstyleliteralemphasis{\sphinxupquote{{[}}}\sphinxhref{https://docs.python.org/3/library/stdtypes.html\#str}{\sphinxstyleliteralemphasis{\sphinxupquote{str}}}\sphinxstyleliteralemphasis{\sphinxupquote{{]}}}\sphinxstyleliteralemphasis{\sphinxupquote{}}) \textendash{} This should be a list of filepaths to assosciated filename list.

\item {} 
\sphinxstyleliteralstrong{\sphinxupquote{filename}} (\sphinxhref{https://docs.python.org/3/library/stdtypes.html\#list}{\sphinxstyleliteralemphasis{\sphinxupquote{list}}}\sphinxstyleliteralemphasis{\sphinxupquote{{[}}}\sphinxhref{https://docs.python.org/3/library/stdtypes.html\#str}{\sphinxstyleliteralemphasis{\sphinxupquote{str}}}\sphinxstyleliteralemphasis{\sphinxupquote{{]}}}\sphinxstyleliteralemphasis{\sphinxupquote{}}) \textendash{} This should be a list of filenames to assosciated filepath (filepath\_list) list.

\item {} 
\sphinxstyleliteralstrong{\sphinxupquote{scale}} (\sphinxhref{https://docs.python.org/3/library/stdtypes.html\#list}{\sphinxstyleliteralemphasis{\sphinxupquote{list}}}\sphinxstyleliteralemphasis{\sphinxupquote{{[}}}\sphinxhref{https://docs.python.org/3/library/functions.html\#float}{\sphinxstyleliteralemphasis{\sphinxupquote{float}}}\sphinxstyleliteralemphasis{\sphinxupquote{{]}}}\sphinxstyleliteralemphasis{\sphinxupquote{}}) \textendash{} This should be a list of scalars to scale assosicated filename dataframe by.

\item {} 
\sphinxstyleliteralstrong{\sphinxupquote{headers\_list}} (\sphinxhref{https://docs.python.org/3/library/stdtypes.html\#list}{\sphinxstyleliteralemphasis{\sphinxupquote{list}}}\sphinxstyleliteralemphasis{\sphinxupquote{{[}}}\sphinxhref{https://docs.python.org/3/library/functions.html\#int}{\sphinxstyleliteralemphasis{\sphinxupquote{int}}}\sphinxstyleliteralemphasis{\sphinxupquote{{]}}}\sphinxstyleliteralemphasis{\sphinxupquote{}}) \textendash{} This should be a list of numbers to exclude number of columns from dataframes.

\item {} 
\sphinxstyleliteralstrong{\sphinxupquote{output\_filename}} (\sphinxstyleliteralemphasis{\sphinxupquote{str.}}) \textendash{} Output filename.

\end{itemize}

\end{description}\end{quote}

\end{fulllineitems}



\chapter{CSV Formatter}
\label{\detokenize{index:csv-formatter}}\label{\detokenize{index:module-CSVFormatter}}\index{CSVFormatter (module)}\index{csvformatter() (in module CSVFormatter)}

\begin{fulllineitems}
\phantomsection\label{\detokenize{index:CSVFormatter.csvformatter}}\pysiglinewithargsret{\sphinxcode{\sphinxupquote{CSVFormatter.}}\sphinxbfcode{\sphinxupquote{csvformatter}}}{\emph{filename}, \emph{olm\_state}, \emph{output\_filename}}{}
This function formats dataset outputs from air quality modelling software to CSV format.
\begin{quote}\begin{description}
\item[{Parameters}] \leavevmode\begin{itemize}
\item {} 
\sphinxstyleliteralstrong{\sphinxupquote{filename}} (\sphinxstyleliteralemphasis{\sphinxupquote{str.}}) \textendash{} This should be a string of the filename to convert to CSV.

\item {} 
\sphinxstyleliteralstrong{\sphinxupquote{olm\_state}} (\sphinxstyleliteralemphasis{\sphinxupquote{bool.}}) \textendash{} This should be a boolean value specifying if the file is OLM format (x and y information header).

\item {} 
\sphinxstyleliteralstrong{\sphinxupquote{output\_filename}} (\sphinxstyleliteralemphasis{\sphinxupquote{str.}}) \textendash{} Output filename.

\end{itemize}

\end{description}\end{quote}

\end{fulllineitems}



\chapter{Factorizer}
\label{\detokenize{index:module-Factorizer}}\label{\detokenize{index:factorizer}}\index{Factorizer (module)}\index{calcrow() (in module Factorizer)}

\begin{fulllineitems}
\phantomsection\label{\detokenize{index:Factorizer.calcrow}}\pysiglinewithargsret{\sphinxcode{\sphinxupquote{Factorizer.}}\sphinxbfcode{\sphinxupquote{calcrow}}}{\emph{row}}{}
This function is used for multiplying entire pandas dataframe single row by scalar value located in the last column of the dataframe.
\begin{quote}\begin{description}
\item[{Parameters}] \leavevmode
\sphinxstyleliteralstrong{\sphinxupquote{row}} (\sphinxstyleliteralemphasis{\sphinxupquote{float.}}) \textendash{} This should be the slice from the pandas dataframe.

\item[{Returns}] \leavevmode
pd.Series \textendash{} The row multiplied by the scalar in the last column of the row

\end{description}\end{quote}

\end{fulllineitems}

\index{factorizer() (in module Factorizer)}

\begin{fulllineitems}
\phantomsection\label{\detokenize{index:Factorizer.factorizer}}\pysiglinewithargsret{\sphinxcode{\sphinxupquote{Factorizer.}}\sphinxbfcode{\sphinxupquote{factorizer}}}{\emph{dataset\_filename}, \emph{factor\_filename}, \emph{output\_filename}}{}
This function mutiplies air quality datasets with vectors given.
\begin{quote}\begin{description}
\item[{Parameters}] \leavevmode\begin{itemize}
\item {} 
\sphinxstyleliteralstrong{\sphinxupquote{dataset\_filename}} (\sphinxstyleliteralemphasis{\sphinxupquote{str.}}) \textendash{} This should be a string of the filename containing the dataset.

\item {} 
\sphinxstyleliteralstrong{\sphinxupquote{factor\_filename}} (\sphinxstyleliteralemphasis{\sphinxupquote{str.}}) \textendash{} This should be a string of the filename containing the vector dataset to mutiply with.

\item {} 
\sphinxstyleliteralstrong{\sphinxupquote{output\_filename}} (\sphinxstyleliteralemphasis{\sphinxupquote{str.}}) \textendash{} Output filename.

\end{itemize}

\end{description}\end{quote}

\end{fulllineitems}



\chapter{NO2Processor}
\label{\detokenize{index:no2processor}}\label{\detokenize{index:module-NO2Processor}}\index{NO2Processor (module)}\index{process() (in module NO2Processor)}

\begin{fulllineitems}
\phantomsection\label{\detokenize{index:NO2Processor.process}}\pysiglinewithargsret{\sphinxcode{\sphinxupquote{NO2Processor.}}\sphinxbfcode{\sphinxupquote{process}}}{\emph{header\_length}, \emph{initial}, \emph{exceedance}, \emph{background\_name}, \emph{input\_data}, \emph{output\_filename}}{}
This function applies air quality modelling functions and generates statistics.
\begin{quote}\begin{description}
\item[{Parameters}] \leavevmode\begin{itemize}
\item {} 
\sphinxstyleliteralstrong{\sphinxupquote{header\_length}} (\sphinxstyleliteralemphasis{\sphinxupquote{int.}}) \textendash{} This should be an integer declaring how many header columns in the dataset

\item {} 
\sphinxstyleliteralstrong{\sphinxupquote{initial}} (\sphinxstyleliteralemphasis{\sphinxupquote{float.}}) \textendash{} This should be an float declaring initial percentage to work with eg 0.1 = 10\%

\item {} 
\sphinxstyleliteralstrong{\sphinxupquote{exceedance}} (\sphinxstyleliteralemphasis{\sphinxupquote{int.}}) \textendash{} This should be an integer declaring how many exceedances eg compare if any are greater than 246

\item {} 
\sphinxstyleliteralstrong{\sphinxupquote{background\_name}} (\sphinxstyleliteralemphasis{\sphinxupquote{str.}}) \textendash{} This should be a string representing input background NO2 filename (Must be located in same directory as .exe).

\item {} 
\sphinxstyleliteralstrong{\sphinxupquote{input\_data}} (\sphinxstyleliteralemphasis{\sphinxupquote{str.}}) \textendash{} This should be a string representing input filename (Must be located in same directory as .exe).

\item {} 
\sphinxstyleliteralstrong{\sphinxupquote{output\_filename}} (\sphinxstyleliteralemphasis{\sphinxupquote{str.}}) \textendash{} Output filename.

\end{itemize}

\end{description}\end{quote}

\end{fulllineitems}



\chapter{Statistics Generator}
\label{\detokenize{index:module-Statistics_Generator}}\label{\detokenize{index:statistics-generator}}\index{Statistics\_Generator (module)}\index{Statistics\_Generator() (in module Statistics\_Generator)}

\begin{fulllineitems}
\phantomsection\label{\detokenize{index:Statistics_Generator.Statistics_Generator}}\pysiglinewithargsret{\sphinxcode{\sphinxupquote{Statistics\_Generator.}}\sphinxbfcode{\sphinxupquote{Statistics\_Generator}}}{\emph{settings}, \emph{header\_length}, \emph{input\_data}, \emph{output\_filename}}{}
This function generates statistics on given input datasets.
\begin{quote}\begin{description}
\item[{Parameters}] \leavevmode\begin{itemize}
\item {} 
\sphinxstyleliteralstrong{\sphinxupquote{settings}} (\sphinxhref{https://docs.python.org/3/library/stdtypes.html\#dict}{\sphinxstyleliteralemphasis{\sphinxupquote{dict}}}\sphinxstyleliteralemphasis{\sphinxupquote{{[}}}\sphinxstyleliteralemphasis{\sphinxupquote{str:int}}\sphinxstyleliteralemphasis{\sphinxupquote{{]}}}\sphinxstyleliteralemphasis{\sphinxupquote{}}) \textendash{} This should be a dictionary of settings with their name as the key and state as value.

\item {} 
\sphinxstyleliteralstrong{\sphinxupquote{header\_length}} (\sphinxstyleliteralemphasis{\sphinxupquote{int.}}) \textendash{} This should be an integer declaring how many header columns in the dataset

\item {} 
\sphinxstyleliteralstrong{\sphinxupquote{input\_data}} (\sphinxstyleliteralemphasis{\sphinxupquote{str.}}) \textendash{} This should be a string representing input filename (Must be located in same directory as .exe).

\item {} 
\sphinxstyleliteralstrong{\sphinxupquote{output\_filename}} (\sphinxstyleliteralemphasis{\sphinxupquote{str.}}) \textendash{} Output filename.

\end{itemize}

\end{description}\end{quote}

\end{fulllineitems}



\renewcommand{\indexname}{Python Module Index}
\begin{sphinxtheindex}
\def\bigletter#1{{\Large\sffamily#1}\nopagebreak\vspace{1mm}}
\bigletter{c}
\item {\sphinxstyleindexentry{CSVFormatter}}\sphinxstyleindexpageref{index:\detokenize{module-CSVFormatter}}
\indexspace
\bigletter{f}
\item {\sphinxstyleindexentry{Factorizer}}\sphinxstyleindexpageref{index:\detokenize{module-Factorizer}}
\indexspace
\bigletter{n}
\item {\sphinxstyleindexentry{NO2Processor}}\sphinxstyleindexpageref{index:\detokenize{module-NO2Processor}}
\indexspace
\bigletter{s}
\item {\sphinxstyleindexentry{Statistics\_Generator}}\sphinxstyleindexpageref{index:\detokenize{module-Statistics_Generator}}
\item {\sphinxstyleindexentry{Stitcher}}\sphinxstyleindexpageref{index:\detokenize{module-Stitcher}}
\end{sphinxtheindex}

\renewcommand{\indexname}{Index}
\printindex
\end{document}